\documentclass[a4paper]{report}

\usepackage[latin1]{inputenc} % accents
\usepackage[T1]{fontenc}      % caract�res fran�ais
\usepackage[francais]{babel}  % langue
\usepackage{graphicx}         % images
\usepackage{verbatim}         % texte pr�format�
\usepackage{listings}
\usepackage[]{algorithm2e}
\usepackage[margin=0.8in]{geometry}
\usepackage[table]{xcolor}

\addto\captionsfrancais{% Replace "english" with the language you use
  \renewcommand{\contentsname}%
    {Sommaire}%
}

\begin{document}

\begin{titlepage}
\fontfamily{phv}\selectfont
\vspace*{\stretch{1}}
\begin{flushright}\LARGE
Accassias
\end{flushright}
\hrule
\begin{flushleft}\huge\bfseries
Developer Manual
\end
{flushleft}
\vspace*{\stretch{2}}
\begin{center}
Eric Therond
\end{center}
\end{titlepage}

\tableofcontents

\chapter{Introduction}

  \section{Pr�sentation d'Accassias}

    \subsection{Description g�n�rale}
\textit{Accassias} est un outil pour l'aide � l'analyse statique de codes sources. Il est �crit dans un but d'apprentissage, � la fois pour l'�tude de la programmation en \textit{C++} et la recherche automatis�e de d�fauts logiciel.

  \newpage

\chapter{Impl�mentation}

  \section{Analyse lexicale}

\begin{tabular}{ | l | l | }
\hline
\rowcolor{gray}Symbole lexical & Chaine \\
\hline
T\_READPOL & t\_readpol \\
\hline
T\_PRINT & t\_print \\
\hline
T\_CLASS & class \\
\hline
T\_NEW & new \\
\hline
T\_FUNCTION & function \\
\hline
T\_IDENTIFIER & ([a-z]\_)* \\
\hline
T\_STRING & ".*" \\
\hline
T\_DECLR & declr \\
\hline
T\_VARIABLE & \$ \\
\hline
T\_NUMBER & [0-9]* \\
\hline
T\_COMMA & , \\
\hline
T\_DOTCOMMA & ; \\
\hline
T\_RIGHTBRACKET & ] \\
\hline
T\_LEFTBRACKET & [ \\
\hline
T\_RIGHTBRACE & \} \\
\hline
T\_LEFTBRACE & \{ \\
\hline
T\_DOUBLEQUOTE & " \\
\hline
T\_RIGHTPARENTHESIS & ) \\
\hline
T\_LEFTPARENTHESIS & ( \\
\hline
T\_EQUAL & = \\
\hline
T\_ADD & + \\
\hline
T\_SUB & - \\
\hline
T\_DIV &  \\
\hline
T\_MUL & * \\
\hline
T\_UNDEF &   \\
\hline
T\_END &   \\
\hline
T\_ERROR &   \\
\hline
T\_RETURN & return \\
\hline
T\_UP\_ARROW &   \\
\hline
T\_DOWN\_ARROW &   \\
\hline
T\_TAB & 	 \\
\hline
T\_IF & if \\
\hline
T\_ELSEIF & elseif \\
\hline
T\_ELSE & else \\
\hline
T\_FOR & for \\
\hline
T\_DO & do \\
\hline
T\_WHILE & while \\
\hline
T\_THIS & this \\
\hline
T\_SUP & > \\
\hline
T\_INF & < \\
\hline
T\_SUPEQUAL & >= \\
\hline
T\_INFEQUAL & <= \\
\hline
T\_SYSTEM & t\_system \\
\hline
T\_FPUTS & t\_fputs \\
\hline
T\_CFG\_DOT & t\_cfg\_dot \\
\hline
T\_CFG\_COMPUTE & t\_cfg\_compute \\
\hline
T\_AST\_DOT & t\_ast\_dot \\
\hline
T\_INCLUDE & t\_include \\
\hline
\end{tabular}

  \section{Analyse syntaxique}

\begin{verbatim}
aca ::= instruction*;

instruction ::= for_instruction | if_instruction | system_instruction 
| readpol_instruction | include_instruction | print_instruction 
| fputs_instruction | cfg_dot_instruction | cfg_compute_instruction
| ast_dot_instruction | return_instruction | call_instruction
| variable | variable_declaration | function_declaration | class_declaration

for_instruction ::= T_FOR T_LEFTPARENTHESIS variable T_DOTCOMMA expression T_DOTCOMMA 
variable T_RIGHTPARENTHESIS bloc_instructions

if_instruction ::= T_IF T_LEFTPARENTHESIS expression T_RIGHTPARENTHESIS bloc_instructions 
(T_ELSEIF T_LEFTPARENTHESIS expression T_RIGHTPARENTHESIS bloc_instructions)* 
(T_ELSE bloc_instructions)?

system_instruction ::= T_SYSTEM

readpol_instruction ::= T_READPOL

include_instruction ::= T_INCLUDE

print_instruction ::= T_PRINT

fputs_instruction ::= T_FPUTS

cfg_dot_instruction ::= T_CFG_DOT

cfg_compute_instruction ::= T_CFG_COMPUTE

ast_dot_instruction ::= T_AST_DOT

return_instruction ::= T_RETURN expression T_DOTCOMMA

call_instruction ::= T_IDENTIFIER T_LEFTPARENTHESIS 
((expression T_COMMA)* (expression T_COMMA))? T_RIGHTPARENTHESIS T_DOTCOMMA

variable ::= T_VARIABLE (T_IDENTIFIER | T_THIS) (T_SUB T_SUP)? 
((T_LEFTBRACKET expression T_RIGHTBRACKET)? variable_affectation) 
| call_instruction

variable_declaration ::= T_DECLR T_VARIABLE T_IDENTIFIER 
(T_LEFTBRACKET expression T_RIGHTBRACKET)? variable_end

function_declaration ::= T_FUNCTION T_IDENTIFIER T_LEFTPARENTHESIS 
(T_DECLR variable_declaration)* (T_RIGHTPARENTHESIS | T_END) bloc_instructions

class_declaration ::= T_CLASS T_IDENTIFIER bloc_classe

expression ::= term ((T_ADD | T_SUB | T_INF | T_INFEQUAL | T_SUP | T_SUPEQUAL) term)?

term ::= factor ((T_MUL | T_DIV) factor)?

factor ::= (T_NUMBER | T_STRING | T_VARIABLE | T_IDENTIFIER | 
(T_LEFTPARENTHESIS expression T_RIGHTPARENTHESIS)

bloc_instructions ::=  T_LEFTBRACE (for_instruction | if_instruction | system_instruction
| readpol_instruction | include_instruction | print_instruction | fputs_instruction
| cfg_dot_instruction | cfg_compute_instruction | ast_dot_instruction | return_instruction
| call_instruction | variable | variable_declaration)* (T_RIGHTBRACE | T_END)

variable_affectation ::= T_EQUAL (T_NEW classe_instance | expression)? variable_end 

variable_end ::=  T_RIGHTPARENTHESIS | T_DOTCOMMA

bloc_classe ::= T_LEFTBRACE (function_declaration | variable_declaration | variable)* 
(T_END | T_RIGHTBRACE)

classe_instance ::= T_IDENTIFIER

\end{verbatim}

\appendix

% \listoffigures
% \listoftables

\end{document}
